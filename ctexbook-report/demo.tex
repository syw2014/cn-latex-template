\documentclass{progartcn}
\usepackage{graphicx}
\usepackage[dvipsnames]{xcolor}
\usepackage{wrapfig}
\usepackage{enumerate}
\usepackage{amsmath,mathrsfs,amsfonts}
\usepackage{booktabs}
\usepackage{tabularx}
\usepackage{colortbl}
\usepackage{multirow,makecell}
\usepackage{multicol}
\usepackage{ulem} % \uline
\usepackage{listings}
\usepackage{tikz}
\usepackage{tcolorbox}
\usepackage{fontawesome}


\title{\bfseries\sffamily
  算法项目代码管理
}
\author{师艳伟}
%\date{}
% \affil{Algorithm Team}


% 导言区结束
\begin{document}
\maketitle      % 制作封面


\sloppy % 解决中英文混排文字超出边界问题


% 添加目录
\tableofcontents
\thispagestyle{empty} % 目录页不显示页码

\mainmatter           % 页码从正文开始
\renewcommand{\thesection}{\arabic{section}}
% \setcounter{section}{2}

% 第0章
\section{版本说明}
% 表1 版本说明
\begin{table}[!ht]
  \centering
  \begin{tabular}{|c|c|c|c|}
  \hline
  文档版本 & 修改说明 & 作者 & 时间	\\
  \hline
  v1.0 & 创建文档 & 师艳伟 & 2018.11.20		\\
  \hline
  \end{tabular}
\end{table}

% 2 chapter
\section{Spark 项目工程模板}
TO BE ADDED!
\subsection{目录结构}
TO BE ADDED!
\subsection{新建应用}
TO BE ADDED!

\section{代码提交说明}

TO BE ADDED!


\section{boxes}

\noindent\verb|\begin{titledbox}{<title>} <content> \end{titledbox}|

\begin{titledbox}{HTTP/Console 内核}
  HTTP 内核继承自 \verb|Illuminate\Foundation\Http\Kernel| 类,该类定义了一个 \verb|bootstrappers| 数组,这个数组中的类在请求被执行前运行,这些 \verb|bootstrappers| 配置了错误处理、日志、检测应用环境以及其它在请求被处理前需要执行的任务。
\end{titledbox}

\noindent\verb|\begin{notebox} <content> \end{notebox}|

\begin{notebox}
  HTTP 内核继承自 \verb|Illuminate\Foundation\Http\Kernel| 类,该类定义了一个 \verb|bootstrappers| 数组,这个数组中的类在请求被执行前运行,这些 \verb|bootstrappers| 配置了错误处理、日志、检测应用环境以及其它在请求被处理前需要执行的任务。
\end{notebox}

\noindent\verb|\begin{importantbox} <content> \end{importantbox}|

\begin{importantbox}
  HTTP 内核继承自 \verb|Illuminate\Foundation\Http\Kernel| 类,该类定义了一个 \verb|bootstrappers| 数组,这个数组中的类在请求被执行前运行,这些 \verb|bootstrappers| 配置了错误处理、日志、检测应用环境以及其它在请求被处理前需要执行的任务。
\end{importantbox}

\noindent\verb|\begin{shellbox} <content> \end{shellbox}|

\begin{shellbox}
cd my-app
ng serve --open
\end{shellbox}

\noindent\verb|\begin{invertedshellbox} <content> \end{invertedshellbox}|

\begin{invertedshellbox}
cd my-app
ng serve --open
\end{invertedshellbox}


\section{服务容器}

Laravel 服务容器是一个用于管理类依赖和执行依赖注入的强大工具。依赖注入听上去很花哨,其实质是通过构造函数或者某些情况下通过 \verb|setter| 方法将类依赖注入到类中。

让我们看一个简单的例子:

\clearpage

\begin{lstlisting}[language=PHP,caption={PHP 代码样例}]
<?php
namespace App\Http\Controllers;

use App\User;
use App\Repositories\UserRepository;
use App\Http\Controllers\Controller;

class UserController extends Controller
{
  /**
  * The user repository implementation.
  *
  * @var UserRepository
  */
  protected $users;
  
  /**
  * Create a new controller instance.
  * 
  * @param  UserRepository  $users
  * @return void
  */
  public function __construct(UserRepository $users)
  {
    $this->users = $users;
  }
  
  ...
}
\end{lstlisting}

深入理解 Laravel 服务容器对于构建功能强大的大型 Laravel 应用而言至关重要,对于贡献代码到 Laravel 核心也很有帮助。

\end{document}
